\documentclass[a4paper,twosides,10pt]{article}

\usepackage[utf8]{inputenc}
\usepackage[english,brazilian]{babel}
\usepackage{indentfirst}
\usepackage{graphicx}
\usepackage{tabularx}
\usepackage{float}
\usepackage{amsmath}
\usepackage{amsfonts}
\usepackage{makeidx}
%\usepackage{siunitx}
\usepackage[squaren]{SIunits}
\usepackage{url}
\usepackage{hyperref}
\usepackage{enumitem}

%\usepackage% [html,3,sections+,next+]
%{tex4ht}

\setlength{\topmargin}{-12.5mm}
\setlength{\headheight}{16pt}
\setlength{\headsep}{10mm}
 \setlength{\textheight}{245mm}

 \setlength{\oddsidemargin}{0mm}
 \setlength{\evensidemargin}{-14mm}
 \setlength{\textwidth}{170mm}
 
\begin{document}

\textbf{Modelo Simplificado da Bomba de Ventilação}

\vspace{1em}

Símbolos:

\begin{itemize}
\item $A_{emb}$: área transversal do embolo;
\item FIO$_2$: fração de O$_2$;
\item $q_{EXP}$: vazão de expiração forçada;
\item $q_{INS}$: vazão de inspiração;
\item $T_{EXPF}$: tempo de expiração forçada;
\item $T_{EXPN}$: tempo de expiração natural;
\item $T_{INS}$: tempo total de inspiração;
\item $V_{olEXP}$: volume de expiração forçada;
\item $V_{olINS}$: volume total de ar/O$_2$ inspirado;
\item $\omega$: velocidade angular do motor;
\item $\omega_{MAX}$: velocidade angular máxima do motor;
\end{itemize}

\begin{equation}
  \label{eq:aemb}
  A_{emb} = \dfrac{\pi}{4}D_{emb}^2
\end{equation}

Posição máxima do cilindro para inspiração:

\begin{equation}
  \label{eq:xt}
  x_T = \dfrac{V_{olINS}}{A_{emb}}
\end{equation}

Posição máxima do cilindro na expiração forçada:

\begin{equation}
  \label{eq:xexp}
  x_{EXP} = \dfrac{q_{EXP}}{A_{emb}k_{pol}} = \dfrac{V_{olEXP}}{T_{EXPF}A_{emb}k_{pol}}
\end{equation}

Constante de transmissão:

\begin{equation}
  \label{eq:kpol}
  k_{pol} = \dfrac{D_2 D}{D_1} \; ,
\end{equation}

\noindent onde $D$ é o diâmetro da polia do motor, $D_1$ é o diâmetro maior da polia da cremalheira e $D_2$ é o diâmetro menor da polia da cremalheira, que fica em contato com ela.

São 5 fases ou estados no total:

\begin{itemize}
\item \textbf{Fase 1}: Deve durar $T_{EXPN}$
  posição final: $x = 0$
  
  $\omega = \omega_{MAX}$

\item \textbf{Fase 2}: de $x = 0$ a $x = x_{O2}$

    $\omega = \omega_{MAX}$

    $x_{O2} = (\mathrm{FIO_2} - 0.2) x_T$

  \item \textbf{Fase 3}: de $x = x_{O2}$ a $x = x_T$

    $\omega = \omega_{MAX}$

  \item \textbf{Fase 4}: de $x = x_T$ a $x = 0$

    $\omega = - \dfrac{q_{INS}}{A_{emb}k_{pol}} = - \dfrac{-V_{olINS}}{T_{INS}A_{emb}k_{pol}}$
    
  \item \textbf{Fase 5}: de $x = 0$ a $x = x_{EXP}$

    $\omega = \dfrac{q_{EXP}}{A_{emb}k_{pol}} = \dfrac{V_{olEXP}}{T_{EXPF}A_{emb}k_{pol}}$
    
\end{itemize}



\end{document}
 

